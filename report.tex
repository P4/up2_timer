\documentclass[a4paper,oneside]{report}

\usepackage[utf8]{inputenc}
\usepackage{polski}

\newcommand{\startstop}{\texttt{START/STOP}}
\newcommand{\reset}{\texttt{RESET}}
\newcommand{\addmin}{\texttt{ADD\textunderscore MIN}}
\newcommand{\addsec}{\texttt{ADD\textunderscore SEC}}

\title{
	\textbf{Technika cyfrowa}
	\\
	Timer (\texttt{MM:SS}) z ustawianiem czasu
	}
\author{
	Paweł Maniecki\\
	Filip Galas
	}
\date{Rok akademicki: 2014/2015}

\begin{document}
\maketitle

\tableofcontents

\chapter{Instrukcja użytkownika}
\section{Opis działania}

Celem projektu jest \emph{timer} zrealizowany na zestawie
edukacyjnym Altera UP2.

Timer wyposażony jest w licznik czasu (\texttt{MM:SS}), którego
stan widoczny jest na wyświetlaczu. Z punktu widzenia użytkownika
układ może znajdować się w jednym z dwóch trybów, pomiędzy którymi
można przestawiać się przyciskiem \startstop :
\begin{enumerate}
\item Tryb \emph{zliczania}.\\
W tym trybie licznik zlicza czas w dół. Działanie przycisków
\reset , \addmin\ i \addsec\ jest zablokowane.
\item Tryb \emph{ustawiania}.\\
W tym trybie zliczanie czasu jest zatrzymane. Możliwe jest
dodawanie minut i/lub sekund do licznika przyciskami \addmin\ i
\addsec\ oraz zresetowanie jego stanu przyciskiem \reset .
\end{enumerate}
Gdy licznik osiągnie stan \texttt{00:00}, układ zasygnalizuje to
miganiem kropek na wyświetlaczu LED. Po wciśnięciu przycisku
\startstop\ timer powróci do trybu \emph{ustawiania}.
\section{Opis interfejsu użytkownika}
% tu można wstawić rysunek układu

\chapter{Realizacja projektu}
\section{Symulacja układu w Multisimie}
\section{Implementacja układu w HDL}

\end{document}
